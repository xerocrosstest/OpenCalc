\documentclass[11pt]{book} 
\usepackage{graphicx}
             % Book class in 11 points
\parindent0pt  \parskip10pt             % make block paragraphs
\raggedright                            % do not right justify

\title{\bf Open Calculus}    % Supply information
\author{Editor-in-chief: Adam Cross}   
          %   for the title page.
\date{\today}                           %   Use current date. 

% Note that book class by default is formatted to be printed back-to-back.
\begin{document}                        % End of preamble, start of text.
\frontmatter                            % only in book class (roman page #s)
\maketitle                              % Print title page.
\tableofcontents                        % Print table of contents
\mainmatter                             % only in book class (arabic page #s)

\part{A Part Heading}                   % Print a "part" heading



\chapter{Limits}     


\section{Exercises}

\begin{enumerate}

\item
Refer to the graphs.

\begin{center}
\includegraphics[width=4in]{limitExerciseImage1.png}
\end{center}

What is $\lim_{x\to 2} f(g(x))$?  What is $\lim_{x\to 2} g(f(x))$?




\item 

We will consider the limit $$\lim_{x\to 0} x^3.$$ 

\begin{enumerate}
\item 
Sketch a graph of $x^3$ around x=0 and draw two horizontal lines representing $\epsilon$ and $-\epsilon$. 



\item \label{b}
 What is the largest (most positive) value x can be so that   $x^3\leq \epsilon$?  
 

 
 \item \label{c}
  What is the most negative value x can be so that $x^3\geq -\epsilon$?



\item
What $\delta$ should we choose so that when $|x-0|<\delta$ we can be sure that $|x^3-0|<\epsilon$?  (Here we write the ``- 0'' part for emphasis so that it looks like the definition of a limit, but it can (and should) be omitted in most writing.  We want $\delta$ small so that $(x-\delta,x+\delta)$ is in the interval you identified in parts \ref{b} and \ref{c} above.



\item
If we require $|x^3-0|<1/1000$, how small does $|x|$ have to be?  



\end{enumerate}

\item

Explain in simple terms why $$\lim_{x\to 0} \frac{1}{x}$$ does not exist.  



\item
Explain in words why $$\lim_{x\to 0} x^2 \sin(2\pi/x)=0$$ (meaning the limit exists and is 0) but the limit $$\lim_{x\to 0} \sin(2\pi/x)$$ does not exist.  You are not required to write a proof, but explain what the essential difference is.









\end{enumerate}




\chapter{Derivatives}  


\chapter{Integrals}  



\end{document}                         